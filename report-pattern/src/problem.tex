\CWHeader{Курсовая работа}

\CWProblem{
Требуется разработать программу, получающую на вход сначала обучающие, а потом тестовые данные. 
При помощи наивного алгоритма Байеса программа должна обучится на первой части данных и классифицировать вторую часть.
    

{\bfseries Формат входных данных:} {Одно из самых частых применений наивного Байесовского классификатора является определение писем
со спамом. Будем считать что 0 класс текста это спам, а 1-не спам. По итогу входные данные будут выглядеть следующим образом:
идут n строк формата <класс>(0 - спам, 1- не спам) <текст письма> (оканчивающийся символом конца строки). После этого следует m
строк вида <текст письма>, каждую из которых надо классифицировать.}

{\bfseries Формат результата:} { Число 0, если тестовое сообщение является спамом или 1, если нет. }
}
\pagebreak
