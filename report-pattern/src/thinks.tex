\section{Выводы}

Выполнив курсовой работу я изучил работу наивного байесовского классификатора и понял, что, несмотря на свою простосту, это может быть мощный инструмент для бинарной классификации данных.
Предположу, что с увеличением классов классификатор будет работать хуже, т.к. вероятности могут стать *ближе* друг к другу, из-за чего сравнивать их будет проблематично. С другой стороны,
формула была переведена в логарифмы, что должно упростить работу с классами в количестве больше двух. В любом случае, чистую формулу напрямую лучше не использовать, т.к. почти мгновенно достигается
арифмитическое переполнение снизу.

\pagebreak
